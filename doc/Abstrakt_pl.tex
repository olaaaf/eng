\chapter*{Streszczenie}

\noindent\textbf{Uczenie ze wzmocnieniem w grze Super Mario Bros.}
Celem pracy inżynierskiej było opracowanie i przetestowanie modelu opartego na uczeniu ze wzmocnieniem w celu przechodzenia poziomów w grze \textit{Super Mario Bros}. Do implementacji wykorzystano algorytm Double Q-Learning, konwolucyjną sieć neuronową oraz narzędzie Weights and Biases do monitorowania eksperymentów. Praca obejmowała analizę wpływu parametrów treningowych, takich jak współczynnik dyskontowy, strategia wygaszania \(\epsilon\), oraz funkcja nagrody, na skuteczność modelu. Wyniki potwierdzają, że odpowiedni dobór parametrów i zrównoważona funkcja nagrody znacząco poprawiają efektywność algorytmu.
