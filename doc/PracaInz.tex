\documentclass{pginz}

%%%%%%%%%Miejsce na dodatkowe pakiety%%%%%%%%%%%%%
\usepackage{subcaption}

\begin{document}

%\includepdf[page={1}]{StronaTytulowa.pdf}
%\includepdf[page={1}]{Oswiadczenie.pdf}
\setcounter{page}{3}

\chapter*{Streszczenie}

\noindent\textbf{Dziedzina nauki i techniki zgodna z OECD} Nauki inżynieryjne i techniczne, Elektrotechnika, elektronika i inżynieria informatyczna, Robotyka i Automatyka


\chapter*{Abstract}

\noindent\textbf{Reinforcement learning for the Super Mario Bros. game}
The aim of this engineering thesis was to develop and evaluate a reinforcement learning-based model capable of completing levels in the \textit{Super Mario Bros} game. The implementation utilized the Double Q-Learning algorithm, a convolutional neural network, and the Weights and Biases tool for experiment monitoring. The thesis analyzed the impact of training parameters, such as the discount factor, epsilon decay strategy, and reward function, on model performance. The results confirm that appropriate parameter tuning and a balanced reward function significantly enhance the algorithm's efficiency.


\tableofcontents
\addcontentsline{toc}{chapter}{Spis treści}

\chapter*{Lista symboli}

\begin{itemize}[noitemsep,topsep=0pt,parsep=0pt,partopsep=0pt,labelwidth=1cm,align=left,itemindent=0pt]
\item[$\mathbf{u}$] - wejście systemu
\item[$\mathbf{Q}_c$] - macierz kowariancji
\item[$F$] - napięcie $\left[ \frac{kg \cdot m}{s^2} \right]$ %\si[per-mode=fraction]{\kilo\gram\meter\per\second\squared}
\item[$R$] - rezystancja [\si{\ohm}]
\end{itemize}
\chapter*{Lista skrótów}

\begin{itemize}[noitemsep,topsep=0pt,parsep=0pt,partopsep=0pt,labelwidth=1cm,align=left,itemindent=0pt]
\item[SD] - Systemy Diagnostyki
\item[EKF] - Rozszerzony filtr Kalmana (ang. \textit{Extended Kalman Filter})
\end{itemize}

\chapter{Wstęp i cel pracy}



\chapter[Przegląd literatury (Adam Nowak)]{PrzeglĄd literatury}
\label{chap:przeglad}
% tu będą kolejne rozdziały

\listoffigures
\addcontentsline{toc}{chapter}{Spis rysunków}
\listoftables
\addcontentsline{toc}{chapter}{Spis tabel}


%alternatywa - bibtex
\begin{thebibliography}{20}
\bibitem{Kow02} Kowalski J., Kabacki J.: Simulation of Network Systems in Education, Proceedings of the XXIV Autumn International Colloquium Advanced Simulation of Systems. ASIS 2002, 9-11 września 2002, Ostrava, Czechy, s. 213-218.
\bibitem{Nat12}	National Center of Biotechnology Information, http://www.ncbi.nlm.nih.gov (data dostępu 20.12.2012 r.).
\bibitem{Now02}	Nowak K.: Dydaktyczny model łączenia sieci LAN za pomocą sieci rozległych. Projekt dyplomowy inżynierski. WETI PG, 2002.
\bibitem{Bar04}	Barzykowski J. i inni: Współczesna metrologia – zagadnienia wybrane. WNT, Warszawa 2004, s. 575.
\end{thebibliography}

%*****************
%wymagane dodatki:
% Opis dyplomu
% Zawartość płyty CD
% Instrukcja dla projektanta
% Instrukcja dla użytkownika
\begin{appendices}
%\chapter[Instrukcja dla użytkownika]{Instrukcja dla u\.Zytkownika}
%\include{AppB}
\end{appendices}
%*****************

\end{document}